\documentclass[12pt, leqno, twoside]{article}
\usepackage{amsmath, amsthm, amsfonts, amssymb, graphicx, color}
\usepackage[notref,notcite]{showkeys}
\usepackage{esint}
\setlength{\topmargin}{5mm}
\setlength{\oddsidemargin}{7mm}
\setlength{\evensidemargin}{7mm}
\textwidth=33cc
\textheight=48cc

\numberwithin{equation}{section}
     \newtheorem{thm}{Theorem}[section]
     \newtheorem{cor}[thm]{Corollary}
     \newtheorem{prop}[thm]{Proposition}
     \newtheorem{lem}[thm]{Lemma}
\theoremstyle{definition}
      \newtheorem{defn}{Definition}[section]
     \newtheorem{exmp}{Example}[section]
\theoremstyle{remark}
     \newtheorem{rem}{Remark}[section]

\renewcommand{\labelenumi}{{\rm(\roman{enumi})}}

\newcommand{\C}{\mathbb{C}}
\newcommand{\N}{\mathbb{N}}
\newcommand{\R}{\mathbb{R}}
\newcommand{\T}{\mathbb{T}}
\newcommand{\Z}{\mathbb{Z}}
\newcommand{\cA}{\mathcal{A}}
\newcommand{\cC}{\mathcal{C}}
\newcommand{\cF}{\mathcal{F}}
\newcommand{\cG}{\mathcal{G}}
\newcommand{\cI}{\mathcal{I}}
\newcommand{\cL}{\mathcal{L}}
\newcommand{\cM}{\mathcal{M}}
\newcommand{\cN}{\mathcal{N}}
\newcommand{\cQ}{\mathcal{Q}}
\newcommand{\cS}{\mathcal{S}}
\newcommand{\cT}{\mathcal{T}}
\newcommand{\cY}{\mathcal{Y}}

\newcommand{\Lip}{\mathrm{Lip}}

\newcommand{\supp}{\operatorname{supp}}
\newcommand{\esssup}{\mathop{\mathrm{ess\,sup}}}
\newcommand{\essinf}{\mathop{\mathrm{ess\,inf}}}

\newcommand{\ls}{\lesssim}
\newcommand{\gs}{\gtrsim}

\newcommand{\Cic}{C^{\infty}_{\comp}}
\newcommand{\Lic}{L^{\infty}_{\comp}}

\newcommand{\dlim}{\displaystyle\lim}
\newcommand{\dint}{\displaystyle\int}
\newcommand{\sgn}{\mathrm{sgn}}

\pagestyle{plain}

\begin{document}

\baselineskip=18pt

\title{
test title
}
\author{test Author}
%\date{}

\maketitle

\begin{abstract}
test abstract
\end{abstract}

%%%===================================================================
%%%===================================================================
\section{Introduction}\label{sec:intro}
%%%===================================================================
%%%===================================================================

Let $\R^n$ be the $n$-dimensional Euclidean space.
For a function $I:(0,\infty)\to(0,\infty)$, let
\begin{equation}\label{test}
	I(x) = \int_{0}^{x} t \,dt.
\end{equation}
\begin{equation*}
a^2 + b^2 = c^2
\end{equation*}


%%%===================================================================
%%%===================================================================
\section{Definitions and the main result}\label{sec:main}
%%%===================================================================
%%%===================================================================

%%%=============
\begin{defn}
%%%=============
$\alpha$: limit value
\[
     \lim_{n \to \infty} a_{n} = \alpha
\]
\end{defn}
%%%=============
\begin{thm}
%%%=============
\[
     \lim_{n \to \infty} a_{n} = \alpha,\,
     \lim_{n \to \infty} b_{n} = \beta,\,\text{ then }
     \lim_{n \to \infty} (a_{n} + b_{n}) = \alpha + \beta
\]
\end{thm}

\end{document}